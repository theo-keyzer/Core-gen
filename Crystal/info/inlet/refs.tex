The \texttt{R1} is a ref to another node, \texttt{F1}, ref to local
node, \texttt{L1}, ref to local node of the \texttt{R1}, M1, ref to
local node of an element of \texttt{R1} of a node that has a \texttt{L1}
ref.

Sample of of a units file.

\begin{verbatim}
----------------------------------------------------------------
Comp Attr parent Type FindIn
----------------------------------------------------------------

    Element table    R1 Type             * Pointer to (Type).
    Element relation C1 WORD             * Relation type
    Element name     C1 NAME             * Colomn name.

Ref table Type check

----------------------------------------------------------------
Comp Where parent Type
----------------------------------------------------------------

    Element attr     F1 Attr      * Field name
    Element from_id  M1 Attr      * From id

Ref     attr Attr check
Ref3 from_id Attr attr Attr table check
\end{verbatim}

The \texttt{Ref3} uses the \texttt{M1} field \texttt{from\_id}. The link
goes to node of type \texttt{Attr}. It uses the \texttt{attr} field (the
one with the \texttt{F1}) to get to \texttt{Attr} node. In that node it
uses the \texttt{table} field (the one with the \texttt{R1}) to be used
as the parent \texttt{(Type)} to find the \texttt{Attr} in. The
\texttt{from\_id,\ attr} can be different node types. The refs run in a
sequence at \texttt{Element} level. First it does the \texttt{F1,\ R1}
ones, then the the \texttt{M1,\ L1} ones.

Sample of def file.

\begin{verbatim}
--------------------------------------------------------
Type User User
--------------------------------------------------------

Attr Contractor_employee     view contractor_name

Where contractor_name  id_number = contractor_id

---------------------------------------------
Type Contractor_employee Contractor Employee
---------------------------------------------

Attr . . id_number
\end{verbatim}

If this data had to be loaded into a database, the foreign links need to
be populated. Need select statements to get to the id's for this. The
loader's refs does this.

\begin{verbatim}
Attr.tablep = Select id from type where name = 'Contractor_employee'
Where.attrp = Select id from attr where name = 'contractor_name' and attr.parentp = parentp
Where.from_idp = Select id from attr where name = 'id_number' and attr.parentp = Atrr[ Where.attrp ].tablep
\end{verbatim}

To use the links from the \texttt{Where} node, use
\texttt{attr,from\_id,id}. To use it from the \texttt{Attr} node, use
\texttt{Where\_attr,\ Where\_from\_id,\ Where\_id}. These are the
reverse links that loops to match. The \texttt{Its} cmd will get them
all, The variables will get the first one.

The \texttt{L1} is a simpler model of this, it uses the \texttt{R1}
instead of the \texttt{F1} to get to the parent. The \texttt{F1} share
the same parent. The \texttt{R1} finds the parent - top level nodes.

The \texttt{check} on the refs, means it is an error if it does find the
link. A \texttt{(.)} here, means it is optional link. The value then
also need to be a \texttt{(.)} if it is optional. If the value if
different, then it is an error if not found. A \texttt{(?)} here means
link if can, but no error if not.

In the \texttt{db/note.unit}, the \texttt{Q1} links to a node where the
node's parent is referenced by a field that is in the link's parent
node.

\begin{verbatim}
----------------------------------------------------------------
Comp Domain parent . Find
----------------------------------------------------------------

    Element name       C1 WORD       * node name
    
----------------------------------------------------------------
Comp Model parent Domain FindIn
----------------------------------------------------------------

    Element name       C1 WORD       * node name

----------------------------------------------------------------
Comp Frame parent Model FindIn
----------------------------------------------------------------

    Element domain     RS WORD       * ref to domain
    Element name       C1 WORD       * frame name
    
Ref domain Domain ?

----------------------------------------------------------------
Comp A parent Frame FindIn
----------------------------------------------------------------

    Element model      QS WORD       * ref to model
    Element name       C1 WORD       * a name

Refq model Model domain Frame ?
\end{verbatim}

The \texttt{Refq} uses the \texttt{Q1} field \texttt{model}. The link
goes to node of type \texttt{Model}. It uses the \texttt{domain} field
(the one with the \texttt{R1}) to be used as the parent
\texttt{(Domain)} to find the \texttt{Model} in. This is the same as
\texttt{L1} except the \texttt{domain} field is not in \texttt{A}, but
in \texttt{A\textquotesingle{}s} parent, \texttt{Frame}.

\begin{verbatim}
----------------------------------------------------------------
Comp Use parent A
----------------------------------------------------------------

    Element frame      QS Frame      * ref to frame
    Element a          L1 A          * ref to a in frame

Refq frame Frame model A ?
Ref2 a A frame ?
\end{verbatim}

The \texttt{Ref2} uses the \texttt{L1} field \texttt{a}. The link goes
to node of type \texttt{A}. It uses the \texttt{frame} field (the one
with the \texttt{Q1}) to be used as the parent (\texttt{Frame}) to find
the \texttt{A} in.

The \texttt{CS,RS,FS,LS,MS,NS,QS} data types is the same as
\texttt{C1,R1,F1,L1,M1,N1,Q1} except the whitespace between the words
can be a \texttt{:}.

\begin{verbatim}
Frame 2 info:docs - making documentation

A   find:info   * find relavant information in a document
\end{verbatim}
