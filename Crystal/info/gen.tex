\documentclass[11pt]{article}
    \addtolength{\topmargin}{-3cm}
    \addtolength{\textheight}{3cm}
\begin{document}
\section{Startup}
Start.
The generator takes a \texttt{,} separated list of actor files followed
by a \texttt{,} separated list of def input files. They each are all
lumped together.

The first actor's name, is the starting actor. The \texttt{go\_act}
function loop through all actors with this name.

All comand line arguments are store in the starting node instance as
named entries. They are \texttt{\$\{0\},\ \$\{1\}} variables. To access
these variables else where, prefix it with the starting actor's name
like \texttt{\$\{.main.1\}}.
\section{Variable}
Variable names.
The \texttt{\$\{name\}} gets replaced by the value of the variable
\texttt{name}. To escape the \texttt{\$\{}, use \texttt{\$\$\{} The case
conversion letter \texttt{c} like \texttt{\$\{name\}c}, captilize the
variable's value. The \texttt{\$\{.arg\}} is the value of the argument
passed from the previous actor. The \texttt{\$\{eval\}\$}, it replaces
the eval with the \texttt{strs} function of the value of \texttt{eval}.
That is to get a code block from a separate file and print it
\texttt{C\ \$\{code\_block\}\$}.
\subsection{Purpose}
The use cases.
\begin{description}
\item[Output]  print variable value.
\item[Match]  compare value.
\end{description}
\subsection{Special}
Special variable names are prefixed by (.).
\begin{description}
\item[Global]  .Node.item.var The item index name of a Node.
\item[Window]  .actor, the def of the actor.
\item[Collections]  .Json, loaded json file.
\item[Counters]  .+, loop counter.
\item[Depth]  .depth, the actor stack depth.
\item[Arg]  .arg, argument passed from previous actor.
\item[Conditional]  .0, first or rest of loop counter.
\item[Eval]  \$, the contence is re evaluated.
\item[Optional]  ?, no error on var.
\end{description}
\subsection{Errors}
Variable name errors.
The errors land up in the generated code to track down the error. Some
commands make use of the \texttt{s\_get\_var,\ strs} functions that
would return the error, but the commands ignore them. The errors are
printed though.
\section{Actor}
The actors
The actor are like functions that can be called and a case like
statement that matches. The match is (var exp string), the string can
have variables in it. Actors of the same name, are the case items. They
are given an input node to operate on. The actor has a list of commands
it runs through.
\subsection{Purpose}
Use cases.
\begin{description}
\item[Navigate]  call actor with a def.
\item[Collect]  collect defs or strings.
\item[Limit]  break out of loops.
\item[Print]  print output text.
\end{description}
\subsection{Name}
Command names.
\begin{description}
\item[All]  actor call with all nodes of type.
\item[Its]  actor call with defs related to current node.
\item[Du]  actor call with the current node.
\item[C]  print output line.
\item[Cs]  print output with no new line.
\item[Unique]  end actor if item is not unique.
\item[Collect]  add def to a collection.
\item[Group]  add strings to a group.
\item[Break]  break out of the actor.
\item[Include]  include file context to output.
\item[Out]  delay or omitting output based on further output.
\item[New]  add new node to input data.
\item[Refs]  run loader refs after adding new nodes.
\item[Var]  create new variable in the current node.
\end{description}
\subsubsection{Var}
Var command.
This creates a named entry in the the current node's instance. The
\texttt{Var\ foo\ =\ bar}, sets the variable. To access it,
\texttt{\$\{foo\}} The \texttt{Var\ .list\_act.foo\ =\ abar}, set the
variable in the node instance that is current in the \texttt{list\_act}.
The current actors are on the stack. To access it,
\texttt{\$\{.list\_act.foo\}}, or when on that node instance (back to
the list\_act), \texttt{\$\{foo\}}.

Also has a \texttt{regex} that can break down the string as named
entries.
\subsubsection{Collect}
Collection commands.
These are to collect data for later use. They denormalizes the input to
be more elegant for the generator. The \texttt{Collect} and
\texttt{Hash}, store the current node's instance. The \texttt{Unique}
and \texttt{Group} store strings - does not duplicate. The \texttt{Hash}
can be accessed as a var \texttt{\$\{.Hash.A.Open.foo\}}, the others by
the \texttt{All} command.
\subsubsection{Break}
Break command.
The actor loop \texttt{go\_act} works like a \texttt{case} switch.
Actors of the same name, are the \texttt{when} cases. The command loop
\texttt{go\_cmds}, loops through the commands in that actor. The
\texttt{All,\ Its} loop, calls the \texttt{go\_act} function in a loop.
The \texttt{Du} command call another actor with \texttt{go\_act}. To
break out of the \texttt{go\_cmds}'s loop, it uses a break to end the
loop. The \texttt{Break\ cmd} command and \texttt{Unique} does that. To
break out of the actor loop, it returns 2. The \texttt{Break\ actor}
does that. That ends up in the \texttt{go\_act}, that end it. To break
out of the \texttt{Its,\ All} loop, it returns 1. The
\texttt{Break\ loop} command does that. The actor loop \texttt{go\_act}
would return 1 if its return is 1. The return 1 will end up in the
calling \texttt{Its,\ All} command that will continue with the
\texttt{go\_cmds}. The generated code loop functions would return the
result if \texttt{!=\ 0}. The other implementations make use of a depth
value that gets inc/dec. It can go back further. The problem is the
\texttt{Du} command \texttt{Do/Jump} that may or may not be between the
loops.
\subsubsection{Condition}
Break condition.
The \texttt{Break\ cmd\ on\_error\ \$\{val\}}, evals the variable string
with the \texttt{strs} function. If the variables are missing, it would
break out of the \texttt{go\_cmds}. The \texttt{no\_error}, would break
when the variables are present - doing checks.
\subsection{Call}
Actor calls.
The \texttt{All,\ Its\ and\ Du} commands, calls the \texttt{new\_act}
function to set up a new actor window on the stack. It passes the match
data \texttt{(variable,\ eq,\ value)} and \texttt{arg} string. The value
is evaluated from the current node instance. The \texttt{Du} command
calls \texttt{go\_act} with the current node instance, the others, the
generated code that call \texttt{go\_act}. The \texttt{go\_act} function
uses the new node instance. The match uses this instance and return if
the match failed. Then it loops through all actors with its given name.
Each of these actors, have there own match data and skips the ones that
do not match.
\subsubsection{Loop}
Loop counter.
The \texttt{All,\ Its} command calls \texttt{new\_act} first that sets
the next actor's counter to -1. The loop calls the \texttt{go\_act}
function, that increments the counter on match. The \texttt{\$\{.-\}} is
the counter value and \texttt{\$\{.+\}}, the counter +1. Also
\texttt{\$\{.0.string\}} for first (if counter is 0) and
\texttt{\$\{.1.string\}} for rest. The value is \texttt{string} The
\texttt{Du} inherits this value.
\subsection{Match}
Actor matching.
Actor have a case like match on all the actors of the same name.
\texttt{Actor\ list\_act\ Node\ name\ =\ tb1}, here it matches the
varable \texttt{name} to \texttt{tb1} The \texttt{\&=} would be false if
the previous one failed. The \texttt{\textbar{}=} would be true if the
previous one was true. The variable has a \texttt{?} option like
\texttt{name?\ =\ tb1}. This would fail if \texttt{name} does not exist.
No error is printed and the global errors flag is not updated - not seen
as an error.
\subsubsection{Matching}
Match cases.
\begin{description}
\item[Equal]  =, var equal to value.
\item[In]  in or has, var is in a list 
\end{description}
\section{Input}
Input files.
\subsection{File}
Input files.
The input files are word based separated by tabs or spaces. The last
column can be a variable string \texttt{(V1)}, that is the string to the
end of the line. There is one whitespace between the previous word and
it. Use a padding word before it to get all the columns alligned if
needed.
\subsection{Other}
Other input.
The Json, Yaml and Xml are addons that operate the same way that the
rest does. May need some more work here.
\subsection{Errors}
Load errors.
The input file loader, prints errors as it goes along, mainly the parent
and refs. The run time only checks these, but does not generate errors.
\section{Window}
Actor stack windows
\subsection{Purpose}
Use cases.
\begin{description}
\item[Store]  stores values needed.
\item[Stack]  window are stored on the calling stack.
\item[Access]  access to stack items.
\end{description}
\subsection{Name}
Window variables.
\begin{description}
\item[name] actor name
\item[cnt] loop counter
\item[dat] node instance
\item[attr] node variable
\item[eq] equation
\item[value] compare value
\item[arg] argument passed from previos actor
\item[flno] line number of the calling actor
\item[is\_on] out delay is on
\item[is\_trig] out delay is triggered
\item[is\_prev] previous actor has trigger
\item[on\_pos] cmd index for trigger
\item[cur\_pos] current cmd index
\item[cur\_act] current actor index
\end{description}
\section{Refs}
refs
The \texttt{R1} is a ref to another node, \texttt{F1}, ref to local
node, \texttt{L1}, ref to local node of the \texttt{R1}, M1, ref to
local node of an element of \texttt{R1} of a node that has a \texttt{L1}
ref.

\begin{verbatim}
----------------------------------------------------------------
Comp Attr parent Type FindIn
----------------------------------------------------------------

    Element table    R1 TABLE            * Pointer to (Table).
    Element name     C1 NAME             * Colomn name.

Ref table Type check

----------------------------------------------------------------
Comp Where parent Type
----------------------------------------------------------------

    Element attr     F1 Attr      * Field name
    Element from_id  M1 Attr      * From id
    Element eq       C1 WORD      * =
    Element id       F1 Attr      * id

Ref     attr Attr check
Ref       id Attr check
Ref3 from_id Attr attr Attr table check
\end{verbatim}

The \texttt{Ref3} uses the \texttt{M1} field \texttt{from\_id}. The link
goes to node of type \texttt{Attr}. It uses the \texttt{attr} field to
get to \texttt{Attr} node. In that node it uses the \texttt{table} field
to be used as the parent \texttt{(Type)} to find the \texttt{Attr} in.
The \texttt{from\_id,\ attr} can be different node types. The refs run
in a sequence at \texttt{Element} level. First it does the
\texttt{F1,\ R1} ones, then the the \texttt{M1,\ L1} ones.

To use the links from the \texttt{Where} node, use
\texttt{attr,from\_id,id}. To use it from the \texttt{Attr} node, use
\texttt{Where\_attr,\ Where\_from\_id,\ Where\_id}. These are the
reverse links that loops to match. The \texttt{Its} cmd will get them
all, The variables will get the first one.

The \texttt{L1} is a simpler model of this, it uses the \texttt{R1}
instead of the \texttt{F1} to get to the parent. The \texttt{F1} share
the same parent. The \texttt{R1} finds the parent - top level nodes.

The \texttt{check} on the refs, means it is an error if it does find the
link. A \texttt{(.)} here, means it is optional link. The value then
also need to be a \texttt{(.)} if it is optional. If the value if
different, then it is an error if not found. A \texttt{(?)} here means
link if can, but no error if not.
\end{document}
