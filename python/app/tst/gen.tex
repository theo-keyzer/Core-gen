\documentclass[11pt]{article}
    \addtolength{\topmargin}{-3cm}
    \addtolength{\textheight}{3cm}
\begin{document}
\section{Startup}
Start.
The generator takes a \texttt{,} separated list of actor files followed
by a \texttt{,} separated list of def input files. They each are all
lumped together.

The first actor's name, is the starting actor. The \texttt{go\_act}
function loop through all actors with this name.

All comand line arguments are store in the starting node instance as
named entries. They are \texttt{\$\{0\},\ \$\{1\}} variables. To access
these variables else where, prefix it with the starting actor's name
like \texttt{\$\{.main.1\}}.

\section{Variable}
Variable names.
The \texttt{\$\{.\_key\}} variable is the value of the key used for when
all key and values are used ( \texttt{This\ list.\ actor} ). The
\texttt{This\ list\ actor} is for all the keys. The value is the list
item of the key The \texttt{This.rev\ list.S\ actor}, the items are
reversed. The \texttt{\$\{.\_set.A\}}, the value is the set item. The
\texttt{\$\{.\_set\}}, the value is the set dict. The
\texttt{\$\{\_.D\}} is better than \texttt{\$\{.\_var.D\}} for some
cases.

The \texttt{strs} function in gen.py`, replaces the variable names of a
string with their values. Some of the actor commands, calls this
function for an item so that the item can be combined with variables.
This is not done for every item, and can be added if needed.

\subsection{Purpose}
The use cases.
\begin{description}
\item[Output]  print variable value.
\item[Match]  compare value.
\end{description}
\subsection{Special}
Special variable names are prefixed by (.).
\begin{description}
\item[Window]  .actor, the def of the actor.
\item[Collections]  .\_set, .\_var, .\_list
\item[Counters]  .+, loop counter.
\item[Depth]  .\_depth, the actor stack depth.
\item[Arg]  .\_arg, argument passed from previous actor.
\item[Conditional]  .0, first or rest of loop counter.
\item[Eval]  \$, the content is re evaluated.
\item[Optional]  ?, no error on var.
\end{description}
\subsection{Errors}
Variable name errors.
The errors land up in the generated code to track down the error. Some
commands make use of the \texttt{s\_get\_var,\ strs} functions that
would return the error, but the commands ignore them. The errors are
printed though.

\section{Actor}
The actors
The actor are like functions that can be called and a case like
statement that matches. The match is (var exp string), the string can
have variables in it. Actors of the same name, are the case items. They
are given an input node to operate on. The actor has a list of commands
it runs through.

The actor match also has a \texttt{?} to match the variable no error.
The \texttt{??}, matches the not found error. There is no error reported
when using it. The error can be on both sides of the equation.
\texttt{name\ =\ \$\{name\}}

\begin{verbatim}
----------------------------------------------------------------
Actor a . model.name ?= test
----------------------------------------------------------------
\end{verbatim}

Here the \texttt{?=} match for if no error.

\begin{verbatim}
----------------------------------------------------------------
Actor a . model.name ??
Break
Actor a . model.name = test
----------------------------------------------------------------
\end{verbatim}

Here the \texttt{Break} will break out of the actor match on error and
not get to next one.

\begin{verbatim}
----------------------------------------------------------------
Actor a . model = test
----------------------------------------------------------------
\end{verbatim}

Or in some cases, this would also work as the values are the same for
both.

The \texttt{?model?} error is no \texttt{model} field where as
\texttt{?model.?}, is no reference to the \texttt{model}.

\subsection{Purpose}
Use cases.
\begin{description}
\item[Navigate]  call actor with a def.
\item[Collect]  collect defs or strings.
\item[Limit]  break out of loops.
\item[Print]  print output text.
\end{description}
\subsection{Name}
Command names.
\begin{description}
\item[All]  actor call with all nodes of type.
\item[Its]  actor call with defs related to current node.
\item[Du]  actor call with the current node.
\item[This]  actor call with data from collections
\item[That]  actor call with nodes from external data like url,db,json,file.
\item[C]  print output line.
\item[Cs]  print output with no new line.
\item[Break]  break out of the actor.
\item[Out]  delay or omitting output based on further output.
\item[Add]  Add to collections
\item[Clear]  Clear collections
\item[Check]  Check unique in collection
\end{description}
\subsubsection{Var}
Var command.
This creates a named entry in the the current node's instance. The
\texttt{Var\ foo\ =\ bar}, sets the variable. To access it,
\texttt{\$\{foo\}} The \texttt{Var\ .list\_act.foo\ =\ abar}, set the
variable in the node instance that is current in the \texttt{list\_act}.
The current actors are on the stack. To access it,
\texttt{\$\{.list\_act.foo\}}, or when on that node instance (back to
the list\_act), \texttt{\$\{foo\}}.

Also has a \texttt{regex} that can break down the string as named
entries.

\subsubsection{Collect}
Collection commands.
\texttt{Add.me\ var\ N} is to add the current node and
\texttt{Add\ var\ Z\ this\ is\ \$\{name\}} to add a string value. To use
it is \texttt{\$\{.\_var.N.name\}} or \texttt{\$\{.\_var.Z\}}.
\texttt{Add.me\ set\ S} and \texttt{Add\ set\ B\ abc} is to is to add to
a set. Sets do not have duplicates. A flag gets set in the window stack
if a duplicate was added. \texttt{Break\ cmds\ for\ .\ True} will end
this actor is the flag is set. \texttt{Check\ set\ B\ abc} does not add,
only checks. The \texttt{Add.break,\ Check.break}, will break the actor
loop like the single \texttt{Break} command. To get more break options,
use a separate \texttt{Break} command. The \texttt{Add\ var} also does a
check to see if the value added is the same. Like
\texttt{Add.break\ var\ done} The \texttt{Add\ me} is to add to the
value to the current node if it is a list,set or dict. The
\texttt{Add.me} is a way to differentiate between using the current node
or the string value. A empty string now no longer defaults to the
current node. The order of the command options does not matter.
\texttt{Add.me.break} is the same as \texttt{Add.break.me} Add list
always adds, but it could break before adding a duplicate. For now, use
the \texttt{Check\ list} for duplicates.

It is now possible now to add to this dict with
\texttt{Add.me\ var\ J.\$\{name\}} The \texttt{me} is the current node
item or it can be a string like
\texttt{Add\ var\ J.\$\{name\}\ \$\{value\}} The
\texttt{Add.node\ var\ J.\$\{name\}\ \_.F}, can add the var F to this
dict. Or \texttt{Add.node\ me\ \$\{name\}\ \_.F} to add the F var node
to the current node. The \texttt{\$\{\_.F\}} is a string whereas
\texttt{\_.F} is the value in it. This to navigate the a node tree, save
it in F with \texttt{Add.me\ var\ F}, then navigate in another node tree
and save it there.

The \texttt{Add\ node:\_.F\ \$\{name\}\ \$\{value\}} is the same as
\texttt{Add\ var\ J.\$\{name\}\ \$\{value\}}

The \texttt{Add.json\ var\ E\ \{"ids":\ {[}4,5,6{]},\ "userId":\ 7\}}
puts a json node in E.

\subsubsection{Break}
Break command.
The \texttt{Break} command is the same as \texttt{Break\ actor} as it is
the default.

The codes returned by the break is 1 for loops, 2 for actor and 3 for
commands. The \texttt{go\_act} function in \texttt{gen.py}, will
continue if the break was for the comands. It will return 0 if its is
for the actor. Else return the value.

The generated code for the \texttt{Its} will continue as long is the
return is 0, else returns the returned value. The commands in the
\texttt{go\_cmd} function that deal with loops, will continue if the
return is for the loops or 0. Else it returns the returned value. There
is no need for a loop continue as a break for the actor will continue
the loop if there was one or continue with the calling actor.

When the \texttt{Break} command specifies the actor the break applies
to, it makes the return value negative and puts a flag on the actor one
up in the calling stack. The actor with the flag on in the
\texttt{go\_act} function will return this value as positive. Then all
the calling code will react in the same way as before. The break is then
for the actor one down.

\texttt{Add.me\ set\ S} and \texttt{Add\ set\ B\ abc} is to is to add to
a set. Sets do not have duplicates. A flag gets set in the window stack
if a duplicate was added. \texttt{Break\ cmds\ for\ .\ True} will end
this actor is the flag is set. \texttt{Check\ set\ B\ abc} does not add,
only checks. The \texttt{Add.break,\ Check.break}, will break the actor
loop like the single \texttt{Break} command. To get more break options,
use a separate \texttt{Break} command. The \texttt{Add\ var} also does a
check to see if the value added is the same. Like
\texttt{Add.break\ var\ done}

\subsubsection{Condition}
Break condition.
\texttt{Break\ cmds\ for\ .\ True} will end this actor is the flag is
set.

The flag is set by the add and check commands.

\subsection{Call}
Actor calls.
The \texttt{All,\ Its\ and\ Du} commands, calls the \texttt{new\_act}
function to set up a new actor window on the stack. It passes the
\texttt{arg} string. The \texttt{Du} command calls \texttt{go\_act} with
the current node instance, the others, the generated code that call
\texttt{go\_act}. The \texttt{go\_act} function uses the new node
instance. The match uses this instance and return if the match failed.
Then it loops through all actors with its given name. Each of these
actors, have there own match data and skips the ones that do not match.

\subsubsection{Loop}
Loop counter.
The \texttt{All,\ Its} command calls \texttt{new\_act} first that sets
the next actor's counter to -1. The loop calls the \texttt{go\_act}
function, that increments the counter on match. The \texttt{\$\{.-\}} is
the counter value and \texttt{\$\{.+\}}, the counter +1. Also
\texttt{\$\{.0.string\}} for first (if counter is 0) and
\texttt{\$\{.1.string\}} for rest. The value is \texttt{string} The
\texttt{Du} inherits this value.

\subsection{Match}
Actor matching.
Actor have a case like match on all the actors of the same name.
\texttt{Actor\ list\_act\ Node\ name\ =\ tb1}, here it matches the
varable \texttt{name} to \texttt{tb1} The \texttt{\&=} would be false if
the previous one failed. The \texttt{\textbar{}=} would be true if the
previous one was true. The variable has a \texttt{?} option like
\texttt{name\ ?=\ tb1}. This would fail if \texttt{name} does not exist.
In this case no error is printed and the global errors flag is not
updated - not seen as an error.

\subsubsection{Matching}
Match cases.
\begin{description}
\item[Equal]  (=), var equal to value.
\item[In]  (in), var is in the value list 
\item[Has]  (has), var list is in the value
\item[Is]  (is), sorted var list is the same as the sorted value list 
\end{description}
\section{Input}
Input files.
\subsection{File}
Input files.
The input files are word based separated by tabs or spaces. The last
column can be a variable string \texttt{(V1)}, that is the string to the
end of the line. There is one whitespace between the previous word and
it. Use a padding word before it to get all the columns alligned if
needed.

\subsection{Other}
Other input.
The Json, Yaml and Xml are addons that operate the same way that the
rest does. May need some more work here.

\subsection{Errors}
Load errors.
The input file loader, prints errors as it goes along, mainly the parent
and refs. The run time only checks these, but does not generate errors.

\subsection{Types}
Data type
\begin{description}
\item[Word] C1, word.
\item[String] V1, string to end of line.
\item[Local] F1, link to local comp - same parent - needs a Ref.
\item[Ref] R1, link to top level comp - Find - needs a Ref.
\item[Indirect] L1, link to child of previous link - R1 for first, L1 for chain - needs a Ref2.
\item[Copy] U0, use a ref field in a node that is refed by a field in the current node - needs a Refu.
\item[Nest] N1, control field of a nested node.
\end{description}
\subsection{Nest}
Node nesting.
A control field of a nested node. The value 1 is for the top level, 2,
next level down and so on. This is to create a tree from one node type.
To navigate to the nodes one level down, use \texttt{Its\ group.right}.
To navigate one level up, use \texttt{Its\ group.left}. The
\texttt{Its\ group.up} goes to the node above it of the same level. The
\texttt{Its\ group.down}, to the node below. The value 0 is for nodes
that do not form part of this set. There can be more than one control
field for different tree layouts.

\begin{verbatim}
----------------------------------------------------------------
Comp Frame parent Model FindIn
----------------------------------------------------------------

    Element group      N1 WORD       * search navigation group index tree
\end{verbatim}

\section{Window}
Actor stack windows
\subsection{Purpose}
Use cases.
\begin{description}
\item[Store]  stores values needed.
\item[Stack]  window are stored on the calling stack.
\item[Access]  access to stack items.
\end{description}
\subsection{Name}
Window variables.
\begin{description}
\item[name] actor name
\item[cnt] loop counter
\item[dat] node instance
\item[attr] node variable
\item[eq] equation
\item[value] compare value
\item[arg] argument passed from previos actor
\item[flno] line number of the calling actor
\item[is\_on] out delay is on
\item[is\_trig] out delay is triggered
\item[is\_prev] previous actor has trigger
\item[on\_pos] cmd index for trigger
\item[cur\_pos] current cmd index
\item[cur\_act] current actor index
\end{description}
\section{Refs}
refs
Knowledge graphs captures information, but may not capture enough detail
how to navigate the graph. The result end up hard codeing the graph's
navigation.

The \texttt{Ref}'s captures the navigation paths while also ensuring the
input is valid.

One input file is used by many actor files to generate even more output
files. So the input needs to be simple for the actors to use and also
have enough detail for the actors to be not hard coded.

The actors also need to be robust enough to deal with input changes. The
input needs to be captued without too much detail.

The core-gen is a boot strap to generate the application generator. For
this it needs the graph diagram of the input. The app generator is then
hard coded to navigate this graph.

For now see the other docs for more detail.

A \texttt{Ref} links a nodes's field to some other node. It can only
link to nodes that do not have a parent (top level nodes). These are
done in the first pass.

The \texttt{Ref2} link to a node by using some other link for the parent
to find the node in it.

The \texttt{Refu} uses a link to a node and copies some other link of
it.

These run in the second pass in the order of the the \texttt{Element}s
of the \texttt{unit} files. Can get a \texttt{not\ resolved} error if
some thing uses a later item. There is a multi pass option
\texttt{refs\_multi\_pass} to solve this.

The \texttt{Refu,Ref2} combination replaces the \texttt{Ref3,Refq} of
other implementations.

\begin{verbatim}
----------------------------------------------------------------
Comp Table parent . Find
----------------------------------------------------------------

    Element name C1 NAME             * Its name.

----------------------------------------------------------------
Comp Attr parent Table FindIn
----------------------------------------------------------------

    Element table    R1 TABLE            * Pointer to (Table).
    Element name     C1 NAME             * Colomn name.

Ref table Table .
\end{verbatim}

On the \texttt{Attr} node, \texttt{Its\ table} is the same as
\texttt{All\ Table} with an actor match of
\texttt{name\ =\ \$\{.prev\_act.table\}} The \texttt{.prev\_act} is any
actor name that is in the calling stack. That actor has the reference to
\texttt{Attr} node that it was working on to get the value of the
\texttt{table} variable.

\begin{verbatim}
----------------------------------------------------------------
Comp Where parent Table
----------------------------------------------------------------

    Element attr     F1 Attr      * Field name
    Element table    U0 Table     * the table of the attr
    Element from_id  L1 Attr      * From id
    Element table2   U0 Table     * the table of the from attr

Ref      attr Attr                           check
Refu    table Table  attr    Attr table
Ref2  from_id Attr   table
Refu   table2 Table  from_id Attr table
\end{verbatim}

On the \texttt{Where} node, \texttt{Its\ attr} is the same as
\texttt{Its\ parent.Attr} with an actor match of
\texttt{name\ =\ \$\{.prev\_act.attr\}}. The \texttt{Its\ table} is the
same as \texttt{Its\ attr.table}. The \texttt{Its\ from\_id} is the same
as \texttt{Its\ table.Attr} with an actor match of
\texttt{name\ =\ \$\{.prev\_act.from\_id\}} The \texttt{Refu} is the
\texttt{attr.table} part and the Ref2, the \texttt{.attr} with the
match. The second \texttt{Refu} chains to another \texttt{table2}. So
\texttt{Its\ table2} is the same as \texttt{Its\ from\_id.table}. So now
another \texttt{Ref2} can go from there.

To go from the \texttt{Attr} node to the \texttt{Where} node,
\texttt{Its\ Where\_attr} is the same as \texttt{Its\ parent.Where} with
actor match \texttt{attr\ =\ \$\{.prev\_act.name\}} and
\texttt{Its\ Where\_from\_id}, the same as \texttt{Its\ parent.Where}
with actor match \texttt{from\_id\ =\ \$\{.prev\_act.name\}}

For refs fields, the variable names work the same as as the \texttt{Its}
command. On the \texttt{Attr} node, it can use
\texttt{\$\{Where\_attr.from\_id.name\}} and
\texttt{\$\{Where\_from\_id.attr.name\}}

The \texttt{Its} command can hadle none to many relations. The variables
will give an error if none, or just use the first one. It asumes you
know wat jou are doing. The variables can't go into child nodes.

\begin{verbatim}
----------------------------------------------------------------
Comp Domain parent . Find
----------------------------------------------------------------

    Element name       C1 WORD       * node name

----------------------------------------------------------------
Comp Model parent Domain FindIn
----------------------------------------------------------------

    Element name       C1 WORD       * node name

----------------------------------------------------------------
Comp Frame parent Model FindIn
----------------------------------------------------------------

    Element group      N1 WORD       * search navigation group index tree
    Element domain     R1 Domain     * ref to domain

Ref domain Domain .

----------------------------------------------------------------
Comp A parent Frame FindIn
----------------------------------------------------------------
* Use domain from parent
* The U0 is a hidden field - only has the pointer
----------------------------------------------------------------

    Element domain     U0 Domain     * the domain of frame
    Element model      L1 Model      * ref to model

Refu domain Domain parent Frame domain .
Ref2 model Model domain .
\end{verbatim}

Here \texttt{Its\ domain} is the same as \texttt{Its\ parent.domain}.
The \texttt{Ref2} will then use the \texttt{domain} to find a
\texttt{model} for it.

The \texttt{group} element with the \texttt{N1} is a nested field. With
the \texttt{Its\ group}, it can get to its sub nodes with a value one
higher than the current one, up to the one with the same level. The
values of zero are skipped. Previous versions had more directions to
navigate.

From the \texttt{gen.unit} file of the base generator

\begin{verbatim}
----------------------------------------------------------------
Comp Comp parent . Find
----------------------------------------------------------------

    Element name   C1 NAME          * of component.
    Element parent R1 COMP          * its parent.

Ref parent Comp .

----------------------------------------------------------------
Comp Element parent Comp FindIn
----------------------------------------------------------------

    Element name C1 NAME  * of element
    Element mw   C1 WORD  * storage type
    Element mw2  C1 WORD  * parser type - not used
    
----------------------------------------------------------------
Comp Ref parent Comp
----------------------------------------------------------------

    Element element F1 ELEMENT       * link to local element
    Element comp    R1 COMP          * link to comp
    Element opt     C1 WORD          * optional or check - error if not found
    
Ref element Element check
Ref    comp Comp    check
    
\end{verbatim}

This works the same way as the app unit files do. On a \texttt{Element}
node it can get a value in a \texttt{Ref} node with
\texttt{\$\{Ref\_element.opt\}} This is because the \texttt{Ref} has a
link to the \texttt{Element} on the \texttt{element} field and this is
just a reverse of it. The \texttt{Element} node could have included
\texttt{comp,opt} and not need the reverse link and just use
\texttt{\$\{opt\}}. But then some of then use \texttt{Ref2,Refu} that
has many more fields that also have to be included. Most of the elements
do not use refs and would make it look messy. A usefull technique to
master.

Added the \texttt{p\_check.act} in \texttt{app/bld} to see if the units
files are correct to some degree. You can build simular ones to see if
the input \texttt{def} files are valid for the actor files that use them
as they just assume it is all correct. It can help with debuging.

The \texttt{Refu,Ref2} are dependent on other relations that may be not
resolved yet. For this it does a multiple passes, but can get stuck on
cirular ones. All references start off as -1. As they get resolved they
change, or go to -2 for no match. A count of all the -1 ones are
returned and when 0, the multiple pass stop. If the count remains the
same between passes, it mean it is stuck and not more passes are needed.
It then does another pass to display the errors. This is only needed for
when a single pass does not work. It is also possible to have some error
in the unit files for an unresolved reference. An unresolved reference
is an error even if no match is not.

To use this option, the \texttt{err\ =\ run.refs(glob.acts)} in main.py
need to be changed to \texttt{err\ =\ run.refs\_multi\_pass(glob.acts)}

The refs have an option flag to specify how to deal with
\texttt{not\ found} errors. If it is \texttt{check}, then it will be an
error. If it is \texttt{?}, then there is no error. Otherwise it is the
optional value to use when none. It is an error if not found and the
value looking for is different to this. It can be \texttt{(.)} or
anything else like \texttt{None}.

\end{document}
