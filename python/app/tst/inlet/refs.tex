Knowledge graphs captures information, but may not capture enough detail
how to navigate the graph. The result end up hard codeing the graph's
navigation.

The \texttt{Ref}'s captures the navigation paths while also ensuring the
input is valid.

One input file is used by many actor files to generate even more output
files. So the input needs to be simple for the actors to use and also
have enough detail for the actors to be not hard coded.

The actors also need to be robust enough to deal with input changes. The
input needs to be captued without too much detail.

The core-gen is a boot strap to generate the application generator. For
this it needs the graph diagram of the input. The app generator is then
hard coded to navigate this graph.

For now see the other docs for more detail.

A \texttt{Ref} links a nodes's field to some other node. It can only
link to nodes that do not have a parent (top level nodes). These are
done in the first pass.

The \texttt{Ref2} link to a node by using some other link for the parent
to find the node in it.

The \texttt{Refu} uses a link to a node and copies some other link of
it.

These run in the second pass in the order of the the \texttt{Element}s
of the \texttt{unit} files. Can get a \texttt{not\ resolved} error if
some thing uses a later item. There is a multi pass option
\texttt{refs\_multi\_pass} to solve this.

The \texttt{Refu,Ref2} combination replaces the \texttt{Ref3,Refq} of
other implementations.

\begin{verbatim}
----------------------------------------------------------------
Comp Table parent . Find
----------------------------------------------------------------

    Element name C1 NAME             * Its name.

----------------------------------------------------------------
Comp Attr parent Table FindIn
----------------------------------------------------------------

    Element table    R1 TABLE            * Pointer to (Table).
    Element name     C1 NAME             * Colomn name.

Ref table Table .
\end{verbatim}

On the \texttt{Attr} node, \texttt{Its\ table} is the same as
\texttt{All\ Table} with an actor match of
\texttt{name\ =\ \$\{.prev\_act.table\}} The \texttt{.prev\_act} is any
actor name that is in the calling stack. That actor has the reference to
\texttt{Attr} node that it was working on to get the value of the
\texttt{table} variable.

\begin{verbatim}
----------------------------------------------------------------
Comp Where parent Table
----------------------------------------------------------------

    Element attr     F1 Attr      * Field name
    Element table    U0 Table     * the table of the attr
    Element from_id  L1 Attr      * From id
    Element table2   U0 Table     * the table of the from attr

Ref      attr Attr                           check
Refu    table Table  attr    Attr table
Ref2  from_id Attr   table
Refu   table2 Table  from_id Attr table
\end{verbatim}

On the \texttt{Where} node, \texttt{Its\ attr} is the same as
\texttt{Its\ parent.Attr} with an actor match of
\texttt{name\ =\ \$\{.prev\_act.attr\}}. The \texttt{Its\ table} is the
same as \texttt{Its\ attr.table}. The \texttt{Its\ from\_id} is the same
as \texttt{Its\ table.Attr} with an actor match of
\texttt{name\ =\ \$\{.prev\_act.from\_id\}} The \texttt{Refu} is the
\texttt{attr.table} part and the Ref2, the \texttt{.attr} with the
match. The second \texttt{Refu} chains to another \texttt{table2}. So
\texttt{Its\ table2} is the same as \texttt{Its\ from\_id.table}. So now
another \texttt{Ref2} can go from there.

To go from the \texttt{Attr} node to the \texttt{Where} node,
\texttt{Its\ Where\_attr} is the same as \texttt{Its\ parent.Where} with
actor match \texttt{attr\ =\ \$\{.prev\_act.name\}} and
\texttt{Its\ Where\_from\_id}, the same as \texttt{Its\ parent.Where}
with actor match \texttt{from\_id\ =\ \$\{.prev\_act.name\}}

For refs fields, the variable names work the same as as the \texttt{Its}
command. On the \texttt{Attr} node, it can use
\texttt{\$\{Where\_attr.from\_id.name\}} and
\texttt{\$\{Where\_from\_id.attr.name\}}

The \texttt{Its} command can hadle none to many relations. The variables
will give an error if none, or just use the first one. It asumes you
know wat jou are doing. The variables can't go into child nodes.

\begin{verbatim}
----------------------------------------------------------------
Comp Domain parent . Find
----------------------------------------------------------------

    Element name       C1 WORD       * node name

----------------------------------------------------------------
Comp Model parent Domain FindIn
----------------------------------------------------------------

    Element name       C1 WORD       * node name

----------------------------------------------------------------
Comp Frame parent Model FindIn
----------------------------------------------------------------

    Element group      N1 WORD       * search navigation group index tree
    Element domain     R1 Domain     * ref to domain

Ref domain Domain .

----------------------------------------------------------------
Comp A parent Frame FindIn
----------------------------------------------------------------
* Use domain from parent
* The U0 is a hidden field - only has the pointer
----------------------------------------------------------------

    Element domain     U0 Domain     * the domain of frame
    Element model      L1 Model      * ref to model

Refu domain Domain parent Frame domain .
Ref2 model Model domain .
\end{verbatim}

Here \texttt{Its\ domain} is the same as \texttt{Its\ parent.domain}.
The \texttt{Ref2} will then use the \texttt{domain} to find a
\texttt{model} for it.

The \texttt{group} element with the \texttt{N1} is a nested field. With
the \texttt{Its\ group}, it can get to its sub nodes with a value one
higher than the current one, up to the one with the same level. The
values of zero are skipped. Previous versions had more directions to
navigate.

From the \texttt{gen.unit} file of the base generator

\begin{verbatim}
----------------------------------------------------------------
Comp Comp parent . Find
----------------------------------------------------------------

    Element name   C1 NAME          * of component.
    Element parent R1 COMP          * its parent.

Ref parent Comp .

----------------------------------------------------------------
Comp Element parent Comp FindIn
----------------------------------------------------------------

    Element name C1 NAME  * of element
    Element mw   C1 WORD  * storage type
    Element mw2  C1 WORD  * parser type - not used
    
----------------------------------------------------------------
Comp Ref parent Comp
----------------------------------------------------------------

    Element element F1 ELEMENT       * link to local element
    Element comp    R1 COMP          * link to comp
    Element opt     C1 WORD          * optional or check - error if not found
    
Ref element Element check
Ref    comp Comp    check
    
\end{verbatim}

This works the same way as the app unit files do. On a \texttt{Element}
node it can get a value in a \texttt{Ref} node with
\texttt{\$\{Ref\_element.opt\}} This is because the \texttt{Ref} has a
link to the \texttt{Element} on the \texttt{element} field and this is
just a reverse of it. The \texttt{Element} node could have included
\texttt{comp,opt} and not need the reverse link and just use
\texttt{\$\{opt\}}. But then some of then use \texttt{Ref2,Refu} that
has many more fields that also have to be included. Most of the elements
do not use refs and would make it look messy. A usefull technique to
master.

Added the \texttt{p\_check.act} in \texttt{app/bld} to see if the units
files are correct to some degree. You can build simular ones to see if
the input \texttt{def} files are valid for the actor files that use them
as they just assume it is all correct. It can help with debuging.

The \texttt{Refu,Ref2} are dependent on other relations that may be not
resolved yet. For this it does a multiple passes, but can get stuck on
cirular ones. All references start off as -1. As they get resolved they
change, or go to -2 for no match. A count of all the -1 ones are
returned and when 0, the multiple pass stop. If the count remains the
same between passes, it mean it is stuck and not more passes are needed.
It then does another pass to display the errors. This is only needed for
when a single pass does not work. It is also possible to have some error
in the unit files for an unresolved reference. An unresolved reference
is an error even if no match is not.

To use this option, the \texttt{err\ =\ run.refs(glob.acts)} in main.py
need to be changed to \texttt{err\ =\ run.refs\_multi\_pass(glob.acts)}

The refs have an option flag to specify how to deal with
\texttt{not\ found} errors. If it is \texttt{check}, then it will be an
error. If it is \texttt{?}, then there is no error. Otherwise it is the
optional value to use when none. It is an error if not found and the
value looking for is different to this. It can be \texttt{(.)} or
anything else like \texttt{None}.
