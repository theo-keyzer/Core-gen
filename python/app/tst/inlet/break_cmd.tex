The \texttt{Break} command is the same as \texttt{Break\ actor} as it is
the default.

The codes returned by the break is 1 for loops, 2 for actor and 3 for
commands. The \texttt{go\_act} function in \texttt{gen.py}, will
continue if the break was for the comands. It will return 0 if its is
for the actor. Else return the value.

The generated code for the \texttt{Its} will continue as long is the
return is 0, else returns the returned value. The commands in the
\texttt{go\_cmd} function that deal with loops, will continue if the
return is for the loops or 0. Else it returns the returned value. There
is no need for a loop continue as a break for the actor will continue
the loop if there was one or continue with the calling actor.

When the \texttt{Break} command specifies the actor the break applies
to, it makes the return value negative and puts a flag on the actor one
up in the calling stack. The actor with the flag on in the
\texttt{go\_act} function will return this value as positive. Then all
the calling code will react in the same way as before. The break is then
for the actor one down.

\texttt{Add.me\ set\ S} and \texttt{Add\ set\ B\ abc} is to is to add to
a set. Sets do not have duplicates. A flag gets set in the window stack
if a duplicate was added. \texttt{Break\ cmds\ for\ .\ True} will end
this actor is the flag is set. \texttt{Check\ set\ B\ abc} does not add,
only checks. The \texttt{Add.break,\ Check.break}, will break the actor
loop like the single \texttt{Break} command. To get more break options,
use a separate \texttt{Break} command. The \texttt{Add\ var} also does a
check to see if the value added is the same. Like
\texttt{Add.break\ var\ done}
